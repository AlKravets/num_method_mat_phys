\documentclass[14pt,a4paper]{scrartcl}
%\usepackage[14pt]{extsizes}
\usepackage[utf8]{inputenc}
\usepackage[english,russian,ukrainian]{babel}
\usepackage{indentfirst}
\usepackage{misccorr}
\usepackage{graphicx}
\usepackage{amsmath}
\usepackage{ upgreek }
\usepackage{ verbatim }

\graphicspath{}
\DeclareGraphicsExtensions{.jpg}

\begin{document}
	\begin{titlepage}
		\begin{center}
			\small{Міністерство освіти та науки України}\\
			\small{Київський національний університет імені Тараса Шевченка}\\
		\end{center}
			\vspace{15em}
		\begin{center}
			\large{Звіт}\\
			\large{До лабораторної роботи №1}\\
			\large{«Розв’язок граничної задачі для звичайного}\\
			\large{диференціального рівняння  другого порядку методом}\\
			\large{Скінченних елементів на базі методу}\\
			\large{Бубнова-Гальоркіна»}\\			
		\end{center}
			
		\vspace{10em}
		

	
		\begin{flushright}
			Студента 4 курсу\\
			Факультету кібернетики\\
			Групи ОМ-4\\
			Кравця Олексія\\
			
		\end{flushright}
		
		\vspace{\fill}

		
		\begin{center}
			\small{Київ, 2019}
		\end{center}
	
	\end{titlepage}


	\newpage

	\section{Постановка Задачі}
	Методом Скінченних елементів розв'язати граничну задачу для звичайного диференціального рівняння другого порядку, та порівняти результат з вже отриманим результатом роботи методу Бубнова-Гальоркіна.

	
	\begin{equation}\label{eq1}
		\begin{cases}
			-\frac{d}{dx}(p(x)\frac{du}{dx})+a(x)\frac{du}{dx}+q(x)u=f(x)  \\ 
			h_1y(0) - h_2y'(0) =0 \\
			H_1y(1) - H_2y'(1) =0 \\
			0<x<1\\
		\end{cases}
	\end{equation}
	
	\begin{gather}
	p(x)= 2-sin(\pi x)\\
	a(x)= sin(\pi x)\\
	q(x)= 5\\
	f(x) = 2x^2 + sin(2x)\\
	h_1 = 0, h_2 = 1\\
	H_1= 1, H_2 = 4
	\end{gather}
	Розв'язати задачу при $N=50, 100$ та порівняти результати з попередньою лабораторною роботою.
	
	\section{Теоретичні відомості}
	Ідея методу схожа з методом Бубнова-Гальоркіна і відрізняється лише базисом. Розв'язок буде мати вигляд:
	
	\begin{equation} \label{eq2}
		u \approx u_N = \sum_{i=0}^{N}c_{i}\phi_{i}
	\end{equation}
	Де $\phi_i$ - базисні функції. Побудуємо їх.
	
	Розділимо відрізок $[0,1]$ на $N$ частин. Отримаємо послідовність $\left\{ x_{i}\right\}_{i=0}^{N}$. Де
	
	\begin{gather} \label{eq3}
			x_{i} = ih, i= \overline{0,N} \\
			h = \frac{1}{N}
	\end{gather}
	Тепер побудуємо функції $\phi_i$
	
	\begin{gather} \label{eq4}
		\phi_{i} = \left\{
			\begin{array}
			[c]{ll}%
				\frac{x-x_{i-1}}{h}, x \in [x_{i-1}, x] \\
				\frac{x_{i+1}-x}{h}, x \in [x_{i}, x_{i+1}]\\			
				0, x \notin [x_{i-1}, x_{i+1}]
			\end{array} 
		\\
		\phi_{0} = \left\{
			\begin{array}
			[c]{ll}%
				0, x \geq x_1 \\
				\frac{x_{i}-x}{h}, 0 \leq x < x_1
			\end{array}
	\end{gather}
	
	Треба задовольнити головним умовам, в нашому випадку це умови першого типу. Відповідно покладемо $c_0 ,c_N =0$, якщо це необхідно.
	
	Помітимо
	
	\begin{equation} \label{eq5}
	\int_{0}^{1} p(x)\phi_{i}'(x)\phi_{j}'(x) \ne 0 
	\end{equation}
	
	лише коли $ \left\{
	\begin{array}{ll}
	i = j+1 \\
	i = j
	\end{array}
$
	
	Тому можемо рахувати інтеграли такого вигляду.
	
	\begin{equation} \label{eq6}
	\int_{x_{i-1}}^{x_{i+1}} p(x)\phi_{i}'(x)\phi_{i \pm 1}'(x)
	\end{equation}
	
	Також можна помітити, що отримана в результаті матриця буде тридіагональною. 


	
	


	
\end{document}